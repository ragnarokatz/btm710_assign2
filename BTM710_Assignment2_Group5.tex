\documentclass{article}

% Language setting
% Replace `english' with e.g. `spanish' to change the document language
\usepackage[english]{babel}

% Set page size and margins
% Replace `letterpaper' with`a4paper' for UK/EU standard size
\usepackage[letterpaper,top=2cm,bottom=2cm,left=3cm,right=3cm,marginparwidth=1.75cm]{geometry}

% Useful packages
\usepackage{amsmath}
\usepackage{graphicx}
\usepackage[colorlinks=true, allcolors=blue]{hyperref}

\title{Assignment 02}
\author{Andy Mai, Md Iztiba, Phu Anh Pham, Bo Wei Yao, TJ LeBlanc}
\date{October 20 2022}

\begin{document}
\maketitle

\section*{Topic 4: Application of Artificial Intelligence and Virtual Reality Technology in the Rehabilitation Training of Track and Field Athletes}

\section*{Abstract}
Track and field athletes often get injured during training and competitions. Rehabilitation training is a method that is widely used to treat these injuries, though the issue is that the rehabilitation process is inefficient. With advances seen in artificial intelligence and virtual reality technology along with how they have been applied in different fields to great effect, integrating them into rehabilitation training could solve its efficiency issues. To confirm if this is the case, analysis was done on causes for common track and field injuries, injury control strategies, and factors that affect the effectiveness of rehabilitation training. An experiment was then conducted where artificial intelligence and virtual reality technologies were used in rehabilitation training. The experiment lasted 6 weeks and was done with 12 athletes split into two groups, with one of the groups using the proposed rehabilitation training system and the other using the traditional system. The results showed that using artificial intelligence and virtual reality technology in rehabilitation training improved athlete physical functions by around 96\%. Based on a 100-point scoring system, the rehabilitation effect seen in athletes that went through the technology assisted training scored an average of 93.79 points while those that didn’t scored a mean of 82.38. The results of this research demonstrate that utilizing artificial intelligence and virtual reality technology in rehabilitation training can significantly improve the recovery times of athletes.
\section*{Purpose of the study}
This study has three purposes: to gain a thorough understanding of common track and field injuries, injury control strategies, and factors that affect the effectiveness of rehabilitation training, to confirm if the usage of artificial intelligence and virtual reality in rehabilitation training can speed up the recovery process in athletes in order to prove the usefulness of combining this type of technology with traditional medical practices \cite{Lai2018}, and to potentially provide new research ideas related to the integration of technology and sports medicine.
\section*{Research methodology}
This paper used quantitative methods as the goal of the study was to examine relationships between several variables. The researchers utilized a variety of techniques, including vector machine and bar chart comparison.
\section*{Research variables}
\begin{itemize}
  \item Body shape
  \item Physical function
  \item Sports quality
  \item Environmental factors
  \item Degree of recovery
\end{itemize}
\section*{Research questions}
\begin{enumerate}
  \item When compared to the old way, did the athlete's body recover from rehabilitation training efficiently?
  \item What effects can virtual reality and artificial intelligence have on rehabilitation training?
  \item Can the use of artificial intelligence and virtual reality technologies provide a full analysis of each athlete's physical variations?
  \item How do factors including adequate preparation, safety of sports equipment, nutritional supplements, and quality training for vulnerable parts contribute to injury prevention measures?
\end{enumerate}
\pagebreak
\section*{Topic 2: Artificial intelligence, digital transformation and cybersecurity in the banking sector: A multi-stakeholder cognition-driven framework}

\section*{Abstract}
The world is becoming more digitized. Every day, more and more companies are adapting to the changing world by transforming their business to be online. These transformations allow companies to engage customers and use technology to conduct business in exciting new ways. One of these institutions is the banking sector which is using a digital transformation to attract potential new customers and keep their current ones more engaged. At the core, a bank's job is to safeguard their customer’s finances, which means users are giving the banks their trust. Banks in turn must keep user trust. Therefore, banks must carefully incorporate and think through cyber security and artificial intelligence challenges throughout their digital transformation and decision making. This study focuses on how to develop a decision-making model for banks which incorporates realism through expert panel advice and cognitive mapping factored in. \\

\section*{Purpose of the study}
Before the study was done, researchers asked the following two questions. Firstly “How are AI, digital transformation and cybersecurity interrelated in the finance and banking sector?” and secondly “How can the dynamics of their cause-and-effect relationships be explored to improve decision-making processes?”. Therefore, it can be seen that this study was done to design a model for the banking sector to evaluate artificial intelligence and cybersecurity when transforming their business into an online age \cite{SHAIKH20171030}. This study will likely also benefit those in branch and corporate levels to formulate strategies for important company decisions \cite{EDEN2004615}. \\
\section*{Research methodology}
This research uses a combination of cognitive mapping and the DEcision MAking Trial and Evaluation Laboratory (DEMATEL) method. Cognitive mapping is a qualitative method, while DEMATEL is a mixed research method. Therefore, when used in combination, this paper uses a mixed method for research.\\

Within a methodological framework based on a constructivist logic, these two tools - cognitive mapping and DEMATEL, formed a holistic, transparent decision-support system \cite{rodrigues2022artificial}.

\section*{Research variables}
\begin{itemize}
  \item Final consumers
  \item Political-legal factors
  \item Employees
  \item Innovation
  \item Internal bank management
\end{itemize}
\section*{Research questions}
\begin{enumerate}
  \item How are AI, digital transformation and cybersecurity interrelated in the finance and banking sector?
  \item How can the dynamics of their cause-and-effect relationships be explored to improve decision-making processes? \cite{rodrigues2022artificial}
  \item From the list of related literature available, such as “Cybersecurity: stakeholder incentives, externalities, and policy options” by Bauer and Van Eeten, what are the research gaps presented in existing studies? \cite{bauer2009cybersecurity}
  \item Based on participants’ values and professional experience and utilizing “post-its technique'' \cite{eden2004cognitive} in the resulting discussion, what key elements are associated with AI, digital transformation and cybersecurity’s ideal combined role in the banking sector? \cite{rodrigues2022artificial}
\end{enumerate}

\bibliography{citation}
%%%%%%%%%%%%%%%%%%%%%%%%% DO NOT CHANGE %%%%%%%%%%%%%%%%%%%%%%%%%%%

\bibliographystyle{plain}
%%%%%%%%%%%%%%%%%%%%% END OF DO NOT CHANGE %%%%%%%%%%%%%%%%%%%%%%%

\end{document}
